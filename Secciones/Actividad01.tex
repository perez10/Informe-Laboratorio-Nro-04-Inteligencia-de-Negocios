\begin{document}

\section{Objetivos}
✓ Realizar el modelamiento dimensional de los ejercicios dados

\section{Requerimientos}
\item{✓ Conocimientos\\
Para el desarrollo de esta práctica se requerirá de los siguientes conocimientos básicos:
- Conocimientos básicos de administración de base de datos Microsoft SQL Server.
- Conocimientos básicos de SQL.\\\\
✓ Software\\
Asimismo se necesita los siguientes aplicativos:
- Microsoft SQL Server 2016 o superior
- Base de datos AdventureWorksDW2016 o superior}

\section{Consideraciones Iniciales}
\item{
✓ Generar todos los modelos fisicos de los diagramas entidad relación y modelo dimensional en bases de datos separadas en Microsoft SQL Server.}

\section{Desarrollo}
\item{
\textbf{Ejercicio N° 01: Envíos}\\\\
El siguiente diagrama E / R simplificado describe el envío de mercancías. Los lotes pertenecientes a ciertos grupos se
envían a ciertos destinos en varios países a través de diferentes modos de transporte. Un cierto centro de costos es
responsable de cada envío. La dimensión de tiempo consiste en mes y año}

\begin{center}
\includegraphics[width=15cm]{./Imagenes/diagrama1}
\end{center}

\begin{center}
\includegraphics[width=15cm]{./Imagenes/dimension1}
\end{center}

\item{
\textbf{Ejercicio N° 02: Reserva de viajes}\\\\
En este esquema de E / R, un cliente (que es de cierto tipo) reserva un viaje en una agencia de viajes. La agencia de viajes trabaja para un determinado operador turístico. El viaje va a un destino determinado que pertenece a un país determinado.La dimensión de tiempo consiste en mes, trimestre y año

\begin{center}
\includegraphics[width=15cm]{./Imagenes/diagrama2}
\end{center}

\begin{center}
\includegraphics[width=15cm]{./Imagenes/dimension2}
\end{center}

\item{
\textbf{Ejercicio N° 03: Gestion de proyectos}\\\\
Este esquema E / R simplificado muestra un caso gestión del proyecto.
El proyecto para un cliente se divide en varios paquetes de trabajo y siempre una persona es responsable de completar la
tarea. Se cuida en un lugar determinado.
La dimensión de tiempo consiste de día, mes y año.

\begin{center}
\includegraphics[width=15cm]{./Imagenes/diagrama3}
\end{center}

\begin{center}
\includegraphics[width=15cm]{./Imagenes/dimension3}
\end{center}

% Bibliografía.
%-----------------------------------------------------------------
\begin{thebibliography}{99}
https://www.isotools.org/2015/02/23/que-es-el-balanced-scorecard-conoce-su-funcionamiento-y-ventajas/\\
https://economipedia.com/definiciones/modelo-canvas.html\\
http://www.infoviews.com.mx/Bitam/ScoreCard/]\\
https://innokabi.com/canvas-de-modelo-de-negocio/\\
https://josefacchin.com/modelo-canvas-de-negocio/\\
https://www.adaptiveus.com/balanced-scorecard-vs-business-model-canvas/\\

\bibitem{Cd94} Autor, \emph{Título}, Revista/Editor, (año)

\end{thebibliography}

\end{document}
